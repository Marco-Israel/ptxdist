
\chapter{Outlook into the feature}%
\label{cha:ausblick}


\chapter{U-Boot working with NFS and TFTP}%
\label{cha:U-Boot working with NFS and TFTP}

For further development it would be useful to get an TFTP and NFS server
running. Therefor an Ethernet interface is needed on an development board but
would decrease the development time significantly.
\\
In this case, the bootloader environment needs to be modified to work with your
tftp-server and your nfs-server. Here are some steps needs to be preformed to
get it running

\begin{listing}[language=bash]
    setenv autoload no
    setenv serverip <serverip>
        # (e.g.: setenv serverip 192.168.100.1)
    setenv ipaddr <ipaddr>
        # (e.g.: setenv ipaddr 192.168.0.10)
    setenv netmask <netmask>
        # (e.g.: setenv netmask 255.255.255.0)
    setenv rootpath <rootpath>
        # (NFS share has to set in /etc/exports on the Computer that
        # runs the NFS server first)
    setenv dtb <dtb>
        # (name of devicetree file to be downloaded from the tftp server)
    setenv kernel <kernel>
        # (name of the Linux kernel image to be downloaded from
        # the tftp server)
\end{listing}

\chapter{QEMU}%
\label{cha:QEMU}
To decrease the development time a split software from hardware bugs,
it would be usefull to simulate the hardware by qemu.
This would make the use of an nfs and tftp server possible easy by adding an
virtual network adapter.
