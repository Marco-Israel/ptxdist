\chapter{Basics}% \label{cha:Basics_and_general_information}
\section{About PTXdist}%
\label{sec:linux_distribution_build}

How to \textbf{install PTXdist}, how to \textbf{use it} and how to to build the
final Linux distribution configured by use use of PTXdist is in detail descried
online on websites. \textbf{This section provides the information where to find
    online descriptions compatible to the tqma335x chip}.


\subsection{PTXdist setup}% \label{sub:PTXdist setup}
To setup PTXdist you need to preform the following steps online documented.

\begin{description}
    \item[To build the Linux distribution] the framework \textit{PTXdist} is
        used. Refer to \newline \textbf{\cite{ptxdist}} to get an introduction
        to PTXdist.
    \item[How to install PTXdist] and get this framework running, is descried in
        \textbf{\cite{tq_bsp119_install}}. You find here how to setup your
        build host (requirements to the building host) and how to install the
        \textbf{right} (compatible) \textbf{cross compiler version}.

        \textbf{ATTENTION:} You should read the \textbf{WARNINGS} from TQ
        carefully to get the right ordering of installation and setup.

        \textbf{ADDITIONALLY} you should take a look into the
        \textbf{\cite[PTXdist installation guide]{ptxdist_install} }
        to get information not described by TQ and to realize what is spacial
        to TQ.\@
    \item[How to setup TQ Systms BSP] for the base chip (like tqma335x)
        is described in short by TQ systems. Refer to
        \textbf{\cite{tq_bsp119_configuration}} and follow the instructions to
        setup the project environment. You need to prefome the steps there for
        BSP Revision up to v117. Here an example of steps to preform true for
        the BSP Version used in this Project.
\end{description}

\subsubsection{Setup for this pre-installed machine }%\label{ssub:this_setup}
In order to setup a new project by using this pre-installed machine with the
same BSP version, here an example of steps to setup the new project as it was
done for this project.

\begin{itemize}
    \item You can call a simple setup script by TQ:\@
        Run from From your <projetroot> (see \fullref{cha:locations})
        \begin{lstlisting}[language=bash, gobble=12,caption={Run TQ-Systems
        enviroment setup script}]
                user@machine: ./tools/config-tqma335x.qt4
        \end{lstlisting}

    \item You also can also manual select a needed configurations as follows:

        \begin{description}
            \item[Select a Software configuration] run from inside <projectroot>

                \begin{lstlisting}[language=bash,gobble=12, caption={Select the
                ptxdist configuraion file},keywordstyle=\color{black}]
            user@machine: ptxdist select ./configs/platforms/tqma335x/mba335x/platformconfig
                \end{lstlisting}~\cite[Selecting a Userland
                Configuration]{ptxdist_manual}

            \item[Select a Hardware configuration] run from inside <projectroot>

                \begin{lstlisting}[language=bash,gobble=12, caption={Select the
                ptxdist configuration file.}]
            user@machine: ptxdist platform ./configs/systems/qt/ptxconfig
                \end{lstlisting}
                You can take a look at~\cite[Selecting
                a Hardware Platform]{ptxdist_manual}
            \item[Select a toolchain]
                The toolchain should be autoselected. If not take do it
                manual
                \begin{lstlisting}[language=bash,gobble=12,caption={Select the
                cross compiler toolchain}]
            user@machine: ptxdist toolchain '/opt/<path-to-your-toolchain/bin'
                \end{lstlisting}
                You can take a look at~\cite[Selecting a Toolchain]
                {ptxdist_manual}
        \end{description}


    \item Finally after preforming one of above steps run a clean default build
        from inside your <projectroot>:
        \begin{lstlisting}[language=bash,gobble=12,caption={Initial PTXdist
        build}]
            user@machine: ptxdist go --git
                #[..downloading and build default packes, kernel bootloader...]
            user@machine: ptxdist images
                #[...building all activated image types ...]
        \end{lstlisting}
\end{itemize}



\subsection{PTXdist usage}%\label{sub:ptxdist_usage}

The online manual~\cite[PTXdist user manual]{ptxdist_manual} descries the basic
usage of PTXdist. Here the chapter \textbf{\#first-steps-with-ptxdist} is for
you of interest. You can also take a look into~\cite{ptxdist_developer} if you
have more special questions, e.g. \textit{like how to reconfigure and recompile
the Linux kernel via PTXdist} or into the~\cite[PTXdist FAQ list]{ptxdist_faq}
(which of course is very short). The most important commands you will need are:

\begin{description}
    \item[Prefome all steps] and finally build the images all in one
        command
        \begin{lstlisting}[language=bash,numbers=none,caption=Build all
        images with PTXdist]
        user@machine: ptxdist images
        \end{lstlisting}
    \item[Deleate a paket]. This is important before you like to rebuild it.
        Especially if you modify on of the configuration files (see next
        command).

        \begin{lstlisting}[language=bash,numbers=none,caption=Deleat a
        builded packet from PTXdsits scope]
        user@machine: ptxdist clean <your-packet>
        \end{lstlisting}
\newpage
    \item[Alter one of the configuration files] (see~\cite[PTXdist user
        manual]{ptxdist_manual}).

        The most important ones are:

        \lstset{style=bash}
        \begin{lstlisting}[language=bash,numbers=none,caption=Alter a
        configuration]
            #PTXdist main configuration
        user@machine: ptxdist menu
            #Project configuration
        user@machine: ptxdist menuconfig
            #Platform configuration
        user@machine: ptxdist platformconfig
            #Linux kernel configuration
        user@machine: ptxdist kernelconfig
        \end{lstlisting}
    \item[(Re-) Compile] a packet see~\cite[PTXdist’s build
        process]{ptxdist_manual}.

        Here an example if you like to extract, compile, configure, patch and
        install a new packed into the targets \gls{rootfs} all in on step you can
        run:

        \begin{lstlisting}[language=bash,numbers=none,caption=Compile and
        install a paket with PTXdist]
        user@machine: ptxdist targetinstall telair-cp_790760_1
        \end{lstlisting}

        Where \textit{ptxdist-cp\_790760\_1} is the \textit{ptxdist rule name}
        describing how to handle packet packets to build. For more details refer
        to~\cite[PTXdist development guide]{ptxdist_developer}.
\end{description}

\subsection{Extends and modify the basic project }%
\label{sub:extends_project}

If you need to extends the current project e.g.\@ by new software packages or an
new project, you should read the following guides:
\begin{description}
    \item[How to extend PTXdist] is descried motley in the
        \textbf{\cite[developer guide]{ptxdist_developer}}. Some extend
        information and \glspl{FAQ} are descried in
        \textbf{\cite{ptxdist_workflows}}, by other websites in the internet
        or if you write your supplier or like TQ Pengutronix an e-mail.

    \item[Background how PTXdist works] is described in the~\cite[PTXdist Users
        Manual]{ptxdist_manual} in the chapter \textbf{\#how-does-it-work}.
\end{description}


\section{Git}%
\label{sec:git}
Git is a distributed (and also central) version control system which only saves
changes to a file instead of copies. Switching between development versions is
easy this way. Additional Git has a lot of other features compared to other
\glspl{VCS} systems. Git was developed 2005 by Linus Torvalds.

Followings are links to tutorials and description which descries what git is,
what its benefits are  and how you can use it.


\begin{description}
    \item[The Git referance] is available in~\cite{gitdoc}. This gives
        an overview to the possible Git command and its meaning.
    \item[An Open PDF book] provided by the git homepage:~\cite{Chacon2014}. It
        also holds an command index on the last side like the online
        documentation mentioned out above. Thanks to PDF and linkage you
        can directly jump to the descriptions with examples and well designed
        pictures. Below is a video series which shows some of the examples and
        pictures inside this book.
    \item[A Video tutorial series] by an English
        speaker:~\cite{gitvideotutorial_en}.
        He orients on the book mentioned out above and gives examples and
        pictures you find inside this books.
    \item[A Video tutorial] (English) by an German
        speaker:~\cite{gitvideotutorial_de_en}
    \item[A Cheat-Sheet] gives an overview with a short descriptions of the most
        used git commands and git commands flags:~\cite{git_cheat_sheet}. You
        can download it as PDF.\@ The website also give some practical examples
        with command line pictures to some of the commands and workflows.
    \item[The Picture of Git states] gives an overview of the local and remote
        repositories philosophic behind git and shows corresponding
        commands~\cite{git_picture}
\end{description}
