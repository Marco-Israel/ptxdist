\chapter{Basics}%


\label{cha:Basics_and_general_information}
\section{About PTXdist}%
\label{sec:linux_distribution_build}

How to \textbf{install PTXdist}, how to \textbf{use it} and how to to build the
final Linux distribution configured by use use of PTXdist is in detail descried
online on websites. \textbf{This section provides the information where to find
    online descriptions compatible to the tqma335x chip}.

\begin{description}
    \item[To build the Linux distribution] the framework \textit{PTXdist} is
        used. Refer to \newline \textbf{\cite{ptxdist}} to get an introduction
        to PTXdist.
    \item[How to install PTXdist] and get this framework running, is descried in
        \textbf{\cite{tq_bsp119_install}}. You find here how to setup your
        build host (requirements to the building host) and how to install the
        \textbf{right} (compatible) \textbf{cross compiler version}.

        \textbf{ATTENTION:} You should read the \textbf{WARNINGS} from TQ
        carefully to get the right ordering of installation and setup.

        \textbf{ADDITIONALLY} you should take a look into the
        \textbf{\cite[PTXdist installation guide]{ptxdist_install} }
        to get information not described by TQ and to realize what is spacial
        to TQ.\@
    \item[How to setup TQ Systms BSP] for the base chip (like tqma335x)
        is described in short by TQ systems. Refer to
        \textbf{\cite{tq_bsp119_configuration}} and follow the instructions to
        setup the project environment.

    \item[How to use PTXdist] you find in the
        \textbf{\cite[PTXdist user manual]{ptxdist_manual}}. Here the chapter
        \textbf{\#first-steps-with-ptxdist} is for you of interest. You can also
        take a look into~\cite{ptxdist_developer} if you have more special
        questions, e.g. \textit{like how to reconfigure and recompile the Linux
        kernel via PTXdist} or into the~\cite[PTXdist FAQ list]{ptxdist_faq}
        (which of course is very short).

       Nevertheless, the most important commands you need are:

\lstset{style=bash}

        \begin{description}
            \item[Prefome all steps] and finally build the images all in one
                command
            \begin{lstlisting}[language=bash,numbers=none,caption=Build all
                images with PTXdist]
                user@machine: ptxdist images
            \end{lstlisting}
        \item[Deleate a paket]. This is important before you like to rebuild it.
            Especially if you modify on of the configuration files (see next
            command).

            \begin{lstlisting}[language=bash,numbers=none,caption=Deleat a
                builded packet from PTXdsits scope]
                user@machine: ptxdist clean <your-packet>
            \end{lstlisting}
        \item[Alter one of the configuration files] (see~\cite[PTXdist user
            manual]{ptxdist_manual}).

            The most important ones are:

            \begin{lstlisting}[language=bash,numbers=none,caption=Alter a
                configuration]
                    #PTXdist main configuration
                user@machine: ptxdist menu
                    #Project configuration
                user@machine: ptxdist menuconfig
                    #Platform configuration
                user@machine: ptxdist platformconfig
                    #Linux kernel configuration
                user@machine: ptxdist kernelconfig
            \end{lstlisting}
        \item[(Re-) Compile] a packet
            \begin{lstlisting}[language=bash,numbers=none,caption=Recompile a
                paket with PTXdist]
                    TODO
            \end{lstlisting}
        \end{description}

    \item[How to extend PTXdist] is descried motley in the
        \textbf{\cite[developer guide]{ptxdist_developer}}. Some extend
        information and \glspl{FAQ} are descried in
        \textbf{\cite{ptxdist_workflows}}, by other websites in the internet
        or if you write your supplier or like TQ Pengutronix an e-mail.

    \item[Background how PTXdist works] is described in the~\cite[PTXdist Users
        Manual]{ptxdist_manual} in the chapter \textbf{\#how-does-it-work}.
\end{description}


\section{Git}%
\label{sec:git}
Git is a distributed (and also central) version control system which only saves
changes to a file instead of copies. Switching between development versions is
easy this way. Additional Git has a lot of other features compared to other
\glspl{VCS} systems. Git was developed 2005 by Linus Torvalds.

Followings are links to tutorials and description which descries what git is,
what its benefits are  and how you can use it.


\begin{description}
    \item[The Git referance] is available in~\cite{gitdoc}. This gives
        an overview to the possible Git command and its meaning.
    \item[An Open PDF book] provided by the git homepage:~\cite{Chacon2014}. It
        also holds an command index on the last side like the online
        documentation mentioned out above. Thanks to PDF and linkage you
        can directly jump to the descriptions with examples and well designed
        pictures. Below is a video series which shows some of the examples and
        pictures inside this book.
    \item[A Video tutorial series] by an English
        speaker:~\cite{gitvideotutorial_en}.
        He orients on the book mentioned out above and gives examples and
        pictures you find inside this books.
    \item[A Video tutorial] (English) by an German
        speaker:~\cite{gitvideotutorial_de_en}
    \item[A Cheat-Sheet] gives an overview with a short descriptions of the most
        used git commands and git commands flags:~\cite{git_cheat_sheet}. You
        can download it as PDF.\@ The website also give some practical examples
        with command line pictures to some of the commands and workflows.
    \item[The Picture of Git states] gives an overview of the local and remote
        repositories philosophic behind git and shows corresponding
        commands~\cite{git_picture}
\end{description}
