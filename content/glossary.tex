% % % % % % % % % % % % % % % % %Glossar
\glsresetall%
\glsunsetall%


%%%%%%%%%%%%%%%%%%%%%%%%%%%%%%%%%%%%%%%%%%%%%%%%%%%%%%%%%%%%%%%%%%%%%%%%%%%%%%%%
% ATTATION:
% - No cite{ref} is allowed here
% - No breaking lines are allowed in the description
%%%%%%%%%%%%%%%%%%%%%%%%%%%%%%%%%%%%%%%%%%%%%%%%%%%%%%%%%%%%%%%%%%%%%%%%%%%%%%%%

%%%%%%%%%%%%%%%%%%%%%%%%%%%%%%%%%%%%%%%%%%%%%%%%%%%%%%%%%%%%%%%%%%%%%%%%%%%%%%%%
%%% Dual Glossar Acronym in once
%%%%%%%%%%%%%%%%%%%%%%%%%%%%%%%%%%%%%%%%%%%%%%%%%%%%%%%%%%%%%%%%%%%%%%%%%%%%%%%%

\newcommand*{\newdualentry}[5][]{%
  \newglossaryentry{main-#2}{name={#4},%
  text={#3\glsadd{#2}},%
  description={{#5}},%
  #1
  }%
  \newglossaryentry{#2}
  {type=\acronymtype,
  first={#4 (#3)},
  name={#3\glsadd{main-#2}},
  description={\glslink{main-#2}{#4}}
  }%
}


\newdualentry{SPL}{SPL}{Second-stage Program Loader}{%
    such as GNU GRUB, BOOTMGR, Syslinux, or the \gls{MLO} are responsible for
    loading eighth the \gls{TSPL} or directly the operating systems which
    finally (re-) initializes the (rest of) the hardware and waits for user
    level operations. If the operating system is loaded directly or in an further
    step depends on the hardware and the chosen bottling process and bootloader.
    They are also called \textbf{Second-stage Boot Loaders}.
}

\newdualentry{FSPL}{FPL}{First Program Loader}{%
    or also called  \textbf{Second-stage Boot Loaders}, is a firmware located in
    ROM on the device and is automatically run on device start-up or after
    \gls{POR}. The ROM bootloader code is hard coded into the device and cannot
    be changed by the user. It initializes a minimal set of hardware and like
    preformed by an BIOS\@. They are also called
    \textbf{First-stage Boot Loaders}.
}

\newdualentry{TSPL}{TPL}{Third-Stage Program Loader}{%
    like barebox, u-boot, GNU GRUB, SysLinux,
    BOOTMGR or is responsible for loading the selected operations system. It
    provides an simple interface to the user, e.g.\@ to select on of different
    different boot locations and operating systems or to preform some basic
    settings from the user give to the operating systems own boot process.
    They are also called  \textbf{Second-stage Boot Loaders},
}

\newdualentry{MLO}{MLO}{Mmc LOader}{%
    is an \gls{SPL}. Please the description there.
}

\newdualentry{PCB}{PCB}{Printed Circuit Board}{%
    is the hardware board without any assembled components. Therefore see
    like \gls{PCBA}.
}

\newdualentry{PCBA}{PCBA}{Printed Circuit Board Assembly}{%
    is a process to mount the SMT and DIP components on the blank \gls{PCB} in
    resulting the final assembled hardware.
}



\newdualentry{hardlink}{hard link}{hard link}{%
    is a link type in Linux/Unix based systems pointing to a file really exists
    at this location. Movements like deleting or moving take directly effect
    on the file. In this way it differs from an Linux/Unix \gls{symlink}. Such
    type of link makes it possible to display a file at different locations but
    having only one file in your file system, aka hard disk. This type of link
    does not in Microsoft Windows systems where you really copy a file resulting
    in two independent files in the filesystem. You only have symbolic links
    here.
}

\newdualentry{symlink}{symbolic link}{symbolic link}{%
    is a link type in Linux/Unix based systems pointing to a file regardless if
    the file exists or not. Movements like deleting or move the link does not
    has any effect on the real file the link pointing to. In this way it differs
    from Linux/Unix an \gls{hardlink}. Symbolic links are similar to links a
    Microsoft Windows based system. Symbolic links are also called
    \gls{softlink}.
}

\newdualentry{VM}{VM}{Virtual Machine}{%
    is a set of simulated computer hardware simulated by an application running
    on real hardware called host system.  Because the simulating application
    need some hardware resources the virtual machine can only be a subset of the
    real hardware. Inside the Virtual Machine an other software can be run
    isolated from the host system.
}

\newdualentry{DRM}{DRM}{Direct Rendering Manager}{%
    is a subsystem of the Linux kernel responsible for interfacing with GPUs of
    modern video cards. DRM exposes an API that user-space programs can use to
    send commands and data to the GPU and perform operations such as configuring
    the mode setting of the display. From~\cite[Direct Rendering Manager]{wikip}
}

\newdualentry{api}{API}{Application Programming Interface}{%
    is a particular set of rules and specifications that a software program can
    follow to access and make use of the services and resources provided by
    another particular software program that implements that API
}

\newdualentry{BSP}{BSP}{Board support packet}{%
    holds \gls{metadata} how to build software
    for a hardware device. So it describes where to download (and which) needed
    software packets, how to build and install them as also configuration files
    and settings. Often it includes pessary compiler packets and other
    stuff needed to successful build software for a device aka hardware board.
}

\newdualentry{NFS}{NFS}{Network File Service}{%
    (also called Network File Service) is a service which provides access to
    data on a host (server) from remote over an Ethernet interface. Thanks to
    the protocol the client thinks the NFS share is a hard disk directly
    connected to the client itself. The client can directly access this data
    without transfer it like \gls{FTP} or \gls{TFTP} before. The real benefits
    is, that the Client always works with an up-to-date version. This means, if
    the Server host is changing a file the changes are directly available to the
    client system. In a Linux system such an NFS can be mounted into the normal
    file system. In such a way it is possible to provide a howl rootfs to a
    system without the need of an own rootfs by the client. This also makes it
    possible that multiple clients can work with the same rootfs provided only
    by a single host.  }


\newdualentry{TFTP}{TFTP}{Trivial File Transfer Protocol}{%
    The \gls{TFTP} is a very simple \gls{FTP} protocol,
    which provides only sending and receiving of data.
}


\newdualentry{FTP}{FTP}{File Transfer Protocol}{%
    A standard network protocol allowing to transfer files from a host to a
    client machine. The Server machine needs to run a compatible FTP Server
    application to provide files to clients and to handle connections from such.
}

\newdualentry{CVS}{CVS}{Concurrent Versions System}{%
    is a free client-server revision control system in the field of
    software development. A version control system keeps track of all work and
    all changes in a set of files, and allows several developers (potentially
    widely separated in space and time) to collaborate. It was initial released
    1990 under the GNU General Public License. Development and support stops
    with a last release in 2008.
}

\newdualentry{VCS}{VCS}{Versions System}{%
    is a component of software configuration management and
	version controlling to track changes on plain text (ASCII) files. Most
	systems like \gls{GIT} provide mechanisms to deal with such changes like
	toggling between one or compare them. \acrlong{VCS} are also known as
    \textit{revision control systems} or \textit{source control systems}.
}


\newdualentry{EOL}{EOL}{End of Line}{%
    is the definition and defined symbol which marks a line as finished. No
    letters coming any more in this line. Having such a marker in a file is
    important for e.g\. an program reading a file to know when it can stops and
    continue the next line. Otherwise it would read wrong or empty characters.
    Beside from that, a programming displaying this file to the new user must
    also know when the line ends and what to show in the next line.
}

\newdualentry{EOF}{EOF}{End of File}{% }
    is  final symbol or symbol in the last line and character~(s) representing
    the end of a file. This is important when reading or representing a file a
    program must know when to stop reading. Otherwise it would read empty lines
    with empty or random characters.
}



\newdualentry{DTB}{DTB}{Device Tree Blob}{%
    is the binary (compiled) form of the \gls{DTS}. This binary gets read
    by the bootloader and the Linux to get information about connected and
    available hardware and where (at which address) they are \textit{mapped} in
    the memory address range.
}

\newdualentry{DTS}{DTS}{Device Tree Source}{%
    is the plain text description of the underlining hardware, about the bus and
    the memory structure. So it describes how a peripheral is connected to the
    CPU and where it is mapped (located) inside the memory address range to get
    access to the registers.  Also the DTS provides input values to the drivers
    like configuration setting, e.g\. a clock and its speed or which interrupts
    are enabled.  The DTS gets compiled to a binary called \gls{DTB}. It is
    compatible with a linker script in bare metal project where no real Linux
    system is running but the hardware must also be configured and made
    available to the software before the application (\textit{main} function)
    starts.
}

\newdualentry{SDK}{SDK}{Software Developing Kit}{%
    is a predefined environment including everything which is need to develop
    and build a software for a specific purpose. All tools, special helper
    scripts and others are already available as builtin or installed
    automatically, configured for the needs of the SDK and ready to use for the
    user.
}

\newdualentry{RootFS}{RootFS}{Root File System}{%
    Is the first (root or entry point) filesystem of a howl Linux file system
    tree.  Into this other destination trees (e.g\.\ folders, drives, shares)
    gets connected (mounted) and are from here available for the users of the
    Linux system. For example the \textit{home, bin, tmp, boot, usr, \ldots}
    trees are located under the RootFS root.
}


%%%%%%%%%%%%%%%%%%%%%%%%%%%%%%%%%%%%%%%%%%%%%%%%%%%%%%%%%%%%%%%%%%%%%%%%%%%%%%%%
%%% Glossar entries only
%%%%%%%%%%%%%%%%%%%%%%%%%%%%%%%%%%%%%%%%%%%%%%%%%%%%%%%%%%%%%%%%%%%%%%%%%%%%%%%%

\newglossaryentry{GIT}{%
name=GIT,
description={%
    Git is a distributed (and also central) version control system which only
    saves changes to a file instead of copies. Switching between development
    versions is easy this way. Additional Git has a lot of other features
    compared to other \glspl{VCS} systems. Git was developed 2005 by Linus
    Torvalds.
    }
}

\newglossaryentry{Dockerfile}{%
name=Dockerfile,
description={A Dockerfile describes how a \gls{Docker} Container (Docker Image)
    should be build and configured and which software should be automatically
    installed and so available inside the Container. It is also possible to copy
    files automatically inside the container or describe how different container
    get connected or Linked.  } }


\newglossaryentry{Docker}{%
    name=Docker,
    description={Docker is a layer based system for vitalization (mostly
        single services or tasks like an Web server). Such \textit{images} are
        described in plain text by a \gls{Dockerfile} recipes and build in a so
        called \textit{container} (an docker Images). This containers can be
        linked together to build a more complex container (on a higher level).
    }
}

\newglossaryentry{QEMU}{%
name=QEMU,
description={QEMU (from „Quick Emulator“) is a free software to visualize
    Hardware like CPU, interfaces or peripherals. It is written in the
    programming language C and is in this way highly extendable (e.g\. by new or
    special peripherals). QEMU makes it possible to run a single software cross
    compiled on the host for a specific target cpu as well as booting howl
    operation systems as an virtual system.
    }
}
\newglossaryentry{CERT}{%
    name=SEI CERT Coding Standards,
    description={is a standards for commonly used programming languages such as
    C, Java, and Perl, and the Android platform. In terms of C and C++ it is
    not so restricted but more flexible than the de-facto standard \gls{MISRA}.
}
}


\newglossaryentry{CWE}{%
    name=CWE,
description={is a community-developed list of common software and hardware
security weaknesses. It serves as a common language, a measuring stick for
security tools, and as a baseline for weakness identification, mitigation, and
prevention efforts.
}
}



\newglossaryentry{MISRA}{%
    name=MISRA,
description={provide world-leading best practice guidelines for the safe and
    secure application of both embedded control systems and standalone software.
    MISRA is a collaboration between manufacturers, component suppliers and
    engineering consultancies which seeks to promote best practice in developing
    safety- and security-related electronic systems and other software-intensive
    applications. To this end MISRA publishes documents that provide accessible
    information for engineers and management, and holds events to permit the
    exchange of experiences between practitioners.
}
}


\newglossaryentry{Yocto}{%
name=Yocto,
description={%
    is a community which provides and extends tools to configure and
    cross-compile howl Linux distributions and software packets directly out of
    the source for a specific target platform. It is oft simple only called
    \textit{Yocto} The main focus of this community is to provide so called
    \textit{recipes} describing how to build software packets which not have any
    dependencies to a specific target like a \gls{mcu}/\gls{cpu} architecture.
    In this way the community distributes an own Linux distribution described by
    recipes and balded by a framework called \textit{\gls{bitbake}}. The minimal
    possible distribution provided by the Yocto community project is called
    \textit{Poky}. This can be expended by a developer (called user) by further
    software packets which on the one hand are also independent to any target
    and mostly provided also by the Yocto community, or on the other hand new
    packets can be target specific. Further on the Yocto community (in
    cooperation with the \gls{OpenEmbedded community}) extends and contributes
    the building framework Bitbake and provides tools for handling the
    distribution build or software development.  The community (as well as the
    OpenEmbedded community) is massively supported by a growing number of well
    known companies mainly focused in any way of embedded systems and
    electronics. The initial goal of this community (and the interest of the
    companies) is to get a howl (standard) working and extendable environment
    running independent from any hardware in am hardware emulator like
    \gls{QEMU} to be a baseline for developing project. That is what Poky is: A
    minimal extendable Linux distribution ready to run in the emulator QEMU
    without dependencies to hardware.
    }
}


\newglossaryentry{Poky}{%
name=Poky,
description={%
    is a minimal extendable Linux distribution ready to run in the
    emulator QEMU without dependencies to hardware. It can be build by the
    framework \gls{bitbake} and is developed and contributed by the
    \textit{\gls{Yocto} community}.
   }
}


\newglossaryentry{OpenEmbedded}{%
name=OpenEmbedded,
description={%
    is a community is focused on porting public target depended software packets
    to be target independent which get public afterwards free for everyone ready
    to use. On the other hand the community develops and provides
    \gls{bitbrecipes} describing how to build such a packet by the framework
    \gls{bitbake}. Last but not least the community develops and contributes the
    as well as Bitbake itself and other helpful tools to develop, (cross-)
    compile and profile (or debug) software. The community (as well as the
    \gls{Yocto} community) is massively supported by a growing number of well
    known companies mainly focused in any way of embedded systems and
    electronics.  } }

\newglossaryentry{bitbrecipes}{%
    name=Bitbake recipes,
    description={%
        are \glspl{metadatafile} describing how to build, configure
        or e.g.\ install a software packet which should be compiled by a
        framework called \gls{bitbake}. Such recipes are mostly a mix of python,
        bash and GNU make.
} }



%\newglossaryentry{Distribution}{%
%    name=Distribution,
%    plural=Distributionen,
%    description={Als Distribution bezeichnet man eine Zusammenstellung von
%        (Software-) Paketen,Versionen,  Konfigurationen, Einstellungen, usw.
%        die als Gesamtpaket veröffentlicht sind oder werden und in sich ohne
%        weiteres Zutun eine definierte Aufgabe erfüllen. Beispielsweise ist
%        ein Betriebssystem eine solche Zusammenstellung das ohne weiters zu tun
%        für einen Satz von Anwendungsfällen genutzt werden kann.
%        } }
%

\newglossaryentry{Feature}{%
name=Feature,
description={A special (helpful) functionality which is not common for such a
    program but often helpful when working with a tool.
    }
}



\newglossaryentry{metadata}{%
name=metadata,
description={Meta data are information about a file. Like the file format, the
    author or how to use the file.}
}


\newglossaryentry{metadatafile}{%
name=metadata files,
description={Meta data files are files holding only \gls{metadata} about one ore
    more other files.
    }
}




%%%%%%%%%%%%%%%%%%%%%%%%%%%%%%%%%%%%%%%%%%%%%%%%%%%%%%%%%%%%%%%%%%%%%%%%%%%%%%%%
%%% Acronym only entries
%%%%%%%%%%%%%%%%%%%%%%%%%%%%%%%%%%%%%%%%%%%%%%%%%%%%%%%%%%%%%%%%%%%%%%%%%%%%%%%%

\newacronym{FAQ}{FAQ}{Frequently Asked Questions}
\newacronym{POR}{POR}{Power on reset}
\newacronym{BIOS}{BIOS}{Basic Input Output}

%    \newacronym{WYSIWYG}{What you see is what you get}
%    \newacronym{LBA} Logical Block Addressing%
%    \newacronym{PDF}{Portable Dokument Format}
%    \newacronym{HFU}{Hochschule Furtwangen}
%    \newacronym{ISO}{International Organization for Standardization}
%    \newacronym{AES}{Advanced Encryption Standard}
%    \newacronym{RC4}{acrfour}
%    \newacronym{ASCII}{American Standard Code for Information Interchange}
%    \newacronym{XRef}{Cross Referenz}
%    \newacronym{HTTP}{Hypertext Transfer Protocol}
%    \newacronym{OLE}{Object Linking and Embedding}
%    \newacronym{EOL}{End of Line}
%    \newacronym{EOF}{End of File}
%    \newacronym{MD5}{Message-Digest Algorithm Version 5}
%    \newacronym{SHA}{Secure Hash Algorithm}
%    \newacronym{ES}{Escaptes Sonderzeichen}
%    \newacronym{SZ}{Sonderzeichen}
%    \newacronym{FDF}{Forms Data Format}
%    \newacronym{PJDF}{Portable Job Ticket Format}
%    \newacronym{HTTP}{Hypertext Transfer Protocol}
%    \newacronym{CMYK}{Cyan, Magenta, Yellow und der Schwarzanteil Kay}
%    \newacronym{RGB}{Rot, Grün und Blau}
%    \newacronym{VDP}{Variable Data Printing}
%    \newacronym{PAdES}{PDF Advanced Electronic Signatures}
%    \newacronym{SP}{Space Character}
%    \newacronym{HT}{Horizontal Tabulator}
%    \newacronym{CR}{Carriage return}
%    \newacronym{LF}{Line feed}
%    \newacronym{FF}{Form feed}
%    \newacronym{NUL}{Null character}
%    \newacronym{OID}{Object Identifier}
%    \newacronym{ETSI}{European Telecommunications Standards Institute}
%    \newacronym{VM}{virtuelle Maschine}
%


