\chapter{Patched and modified files}%
\label{cha:Patched and modified files}
There are a lot of patches to the Linux kernel, bootloader and other software
components which are already provided and integrated into PTXdist by TQ Systems
in the used BSP \cite{BSP_119}. This should work out of the box. No chances are
needed here so this patches are not described here.
\\
Nevertheless some additional patches are needed to get the software and the
hardware running used by the Telair International GmbH. This is what this
chapter describes - the patches added by Telair International GmbH
additionally to the TQ systems patches.


\section{Where are the patches located in PTXdist?}%
\label{sec:Where are the patches located in PTXdist}
\begin{itemize}
    \item Navigate to <your projectroot>/patches.
\end{itemize}

\begin{lstlisting}[language=bash]
    $ user@machine: cd /ptxdist/projects/cp790760/patches
\end{lstlisting}

There you find folders corresponding to the software packets to patch. Patches
done by Telair International GmbH are located inside this packets to patch in a
separated folder called \textit{telair}. They are activated and included into
the build process via the \textit{series.<platform>} file activating both TQ
and Telair patches.
\\
For example patches corresponding to the Linux kernel:
\begin{lstlisting}[language=bash]
    $ user@machine: ls -la /ptxdist/projects/cp790760/patches/linux-4.4/telair
    $ user@machine: cat /ptxdist/projects/cp790760/patches/linux-4.4/series.MBa335x
\end{lstlisting}
\\
To generate the Telair pataches a \textit{new} (the modified version)
and a \textit{old} (the original) version of a (source) file is needed. Both
versions can be found in (see \fullref{howto_patch})
\begin{itemize}
    \item <your projectroot>/local_sources/telair-patches/<sw-packet>/
\end{itemize}

Her you find the mentioned new and old folder holding the tree in the project
to the files. This is importent when using the tools \textit{diff} and
\textit{patch}. Navigate to \fullref{howto_patch} to learn how to use both
tools.



\section{How to create a patch and integrate it into PTXdist?}%
\label{sec:?}

Patches and modified files by Telair International GmbH are unter GIT version
control in an local repository. This documents what is done additionally
by Telair to the clean BSP from TQ. The repository is not uploaded yet outside
the virtualBox. Not or not manual modified files are in the .gitignore list an
so not under local version control. How to use GIT version control refer to
\cite[GIT howto]{git_howto}. To get more information about whats GIT and why to
use it, refer to \cite[GIT Homepage]{git_homepage}.










\section{Patching files}%
\label{sec:Pached files}

\subsection{How to patch files}%
\label{sub:howto_patch}
\begin{description}
    \item[Patching process in PTX dist]
        To learn how patching process work in PTXdist refer to \cite{todo}
    \item[GNU-Tools \textit{diff},  \textit{patch} \textit{merge}] To get an
        introduction of the general Linux patching process using
        standard GNU-Tools, take a look at
        \cite[GNU Diff and Patch]{GNUPatchTools:Patch_Diff_Exaples} and
        \cite[GNU Merge]{GNUPatchTools:Merge_Examples}. Also take a look at the
        Linux manual pages in section 1:
        \begin{itemize}
            \item man 1 patch
            \item man 1 diff
            \item man 1 merge
        \end{itemize}
\end{description}

\subsection{Patched files}%
\label{sub:Patched fiels}
The following files gets influenced (patched) during due to .patch files.

\begin{itemize}
    \item --- currently none ---
\end{itemize}


\section{Manuel modified files}%
\label{sec:Manuel modified files}

The following files are manual modified in the way described below:.

\begin{description}
    \item[rules/mmc-utils.make]
        \begin{itemize}
            \item Original part
                \begin{lstlisting}[caption={rules/mmc-utils}]
 19 #<<<< OLD VERSION
 20 ## No tags: use a fake descriptive commit-ish to include the date
 21 #MMC_UTILS_VERSION  := 2016-06-07-g0ca049f25191
 24 #MMC_UTILS_SUFFIX   := tar.gz
 25 #MMC_UTILS_URL      := git://git.kernel.org/pub/scm/linux/kernel/   	\
								git/cjb/mmc-utils.git;tag=$(MMC_UTILS_VERSION)
 26 #MMC_UTILS_SOURCE   := $(SRCDIR)/$(MMC_UTILS).$(MMC_UTILS_SUFFIX)
 27 #MMC_UTILS_DIR      := $(BUILDDIR)/$(MMC_UTILS)
 28 #MMC_UTILS_LICENSE  := GPL-2.0 AND BSD-3-Clause
 29 #======
                \end{lstlisting}
            \item Modified part

                \begin{lstlisting}[caption={rules/mmc-utils}]
30 #>>>> NEW VERSION
 31 # No tags: use a fake descriptive commit-ish to include the date
 32 MMC_UTILS_VERSION   := 37c86e60c0442fef570b75cd81aeb1db4d0cbaf
 33 #MMC_UTILS_MD5      := da395e908be7e11bd8417cc02485be3
 34 MMC_UTILS_MD5       := 1223011c75169739b73c01caab281e1
 35 MMC_UTILS           := mmc-utils-$(MMC_UTILS_VERSION)
 36 MMC_UTILS_SUFFIX    := tar.gz
 37 MMC_UTILS_URL       := http://sources.buildroot.net/mmc-utils/mmc-utils-$(MMC_UTILS_URL).$(MMC_UTILS_SUFFIX)
 38 MMC_UTILS_SOURCE    := $(SRCDIR)/$(MMC_UTILS).$(MMC_UTILS_SUFFIX)
 39 MMC_UTILS_DIR       := $(BUILDDIR)/$(MMC_UTILS)
 40 MMC_UTILS_LICENSE   := GPL-2.0 AND BSD-3-Clause
                \end{lstlisting}

        \end{itemize}
\end{description}

