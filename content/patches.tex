\chapter{Patched and modified files}%
\label{cha:patches}
Inside the \textit{patches} folder you find different sub folders who's names
are similar to different software packets, Linux kernel or bootloaders
(\fullref{cha:locations}).

Inside this folders you find (in further subfolders\footnotemark[1]) so called
\textit{patch} files holding changes to software source files. Depending on the
software packet you find inside this, folders from TQ Systems (corresponding to
different hardware boards by TQ Systems) as also patches added by Telair
International GmbH.


\section{Patches by TQ Systems}%
\label{sec:patches}
Most of patches are provided and integrated into PTXdist by TQ Systems
coming with the BSP used in this project.~\cite{tq_bsp119}

This should work out of the box. No chances are needed here so this patches are
not described here.

\section{Patches by Telair International GmbH}%
\label{sec:Patch_by_telair} Some patches are needed to get the software and the
hardware running, used by the Telair International GmbH and were not
pre-installed by TQ Systems GmbH. This is what this chapter describes:\@ The
patches added by Telair International GmbH additionally to the TQ systems
patches.

Navigating to the \textit{patches}~\footnotemark[1] folder you find sub folders
corresponding to the software packets to patch. Patches done by Telair
International GmbH are located inside this packets to patch, in a separated
folder called \textit{telair}.

\textbf{NOTE:} Its important to say, they are and \textbf{must be activated}
(included into the build process). Therefore read
Subsection~\ref{sec:integrate_patch}.

\subsection{Software packets patched by Telair}
The following lists packets sub folders inside
\textit{patches}~\footnotemark[1],
holding patch files from Telair International GmbH in a folder called
\textit{telair}:

\begin{itemize}
    \item
\end{itemize}

For example patches by Telair International GmbH to the Linux Kernel (most of he
patches) can be found like:

\begin{lstlisting}[keywordstyle=\color{black},language=bash,caption={Example: Location of an telair patch}]
    $ user@machine: ls -la /ptxdist/projects/cp790760/patches/linux-4.4/telair
    $ user@machine: cat /ptxdist/projects/cp790760/patches/linux-4.4/series.MBa335x
\end{lstlisting}

\subsection{Updated software sources}
To generate a so called \textit{patch} an \textit{new} (the modified version)
and a \textit{old} (the original) version of a (source) file is needed. Both
versions can be found in the \textit{telair-patches} folder itself located in
\textit{<projectroot>/local\_sources}~\footnotemark[3]. Here you find subfolers named by
software packets to change itself holding respectively the new and old
version folders:\@
\begin{itemize}
    \item <your projectroot>/local\_sources/telair-patches/<sw-packet>/old
    \item <your projectroot>/local\_sources/telair-patches/<sw-packet>/new
\end{itemize}


\section{How to create an patch file; how to integrate a patch file into PTXdist}% \label{sec:new_patch}

The following is a listing on descriptions to need tools and its usage.

\begin{description}
    \item[Patching process in PTX dist:]
        To learn how patching process works in PTXdist (during build), refer
        to~\cite{ptxdist_manual}
    \item[GNU-Tools \textit{diff},  \textit{patch} \textit{merge}:] To get an
        introduction of the general Linux patching workflow by using
        standard GNU-Tools, take a look at
        \begin{itemize}
            \item~\cite[GNU Diff and Patch]{GNGNUUpatchDiff}
            \item~\cite[GNU Merge]{GNUmerge}
            \item Linux manual pages in section 1:
                \begin{itemize}
                    \item man 1 patch
                    \item man 1 diff
                    \item man 1 merge
                \end{itemize}
        \end{itemize}
\end{description}


\subsection{How to integrate a patch file into PTXdist}%
\label{sec:integrate_patch}
To \textit{activate} (included into the build process) a patch, PTXdist needs
informed about this via adding the patch to the \textit{series.<platform>}
file located inside each folder of <patches>~\footnotemark[2]. This file
activates globally both, the TQ Systems and Telair International patches.

\textbf{NOTE:} The order is similar the order PTXdist
will preformed the patches. For some software packets the order is important!



\subsection{Manuel modified files}%
\label{sec:manuel_modified_files}

Because PTXdist internal rule files can not patched during build easy, the
following files were directly modified without use of patch mechanism. Also the
Telair Software itself were modified directly. The original version till exists
in the Telair International GmbH \gls{CVS} system. (Refer to
\fullref{cha:software} to get more details about direct preformed software changes).

\lstset{style=verba}

\begin{description}
    \item[rules/mmc-utils.make]
        \begin{itemize}
            \item Original part
\begin{lstlisting}[caption={Original rules/mmc-utils},frame=single,
commentstyle=\color{black}, postbreak=\mbox{\textcolor{red}{$\hookrightarrow$}\space}]
 19 #<<<< OLD VERSION
 20 ## No tags: use a fake descriptive commit-ish to include the date
 21 #MMC_UTILS_VERSION  := 2016-06-07-g0ca049f25191
 24 #MMC_UTILS_SUFFIX   := tar.gz
 25 #MMC_UTILS_URL      := git://git.kernel.org/pub/scm/linux/kernel/   	\
								git/cjb/mmc-utils.git;tag=$(MMC_UTILS_VERSION)
 26 #MMC_UTILS_SOURCE   := $(SRCDIR)/$(MMC_UTILS).$(MMC_UTILS_SUFFIX)
 27 #MMC_UTILS_DIR      := $(BUILDDIR)/$(MMC_UTILS)
 28 #MMC_UTILS_LICENSE  := GPL-2.0 AND BSD-3-Clause
 29 #======
                \end{lstlisting}
            \item Modified part
                \begin{lstlisting}[caption={Modified
                rules/mmc-utils},commentstyle=\color{black},frame=single,
                postbreak=\mbox{\textcolor{red}{$\hookrightarrow$}\space}]
30 #>>>> NEW VERSION
 31 # No tags: use a fake descriptive commit-ish to include the date
 32 MMC_UTILS_VERSION   := 37c86e60c0442fef570b75cd81aeb1db4d0cbaf
 33 #MMC_UTILS_MD5      := da395e908be7e11bd8417cc02485be3
 34 MMC_UTILS_MD5       := 1223011c75169739b73c01caab281e1
 35 MMC_UTILS           := mmc-utils-$(MMC_UTILS_VERSION)
 36 MMC_UTILS_SUFFIX    := tar.gz
 37 MMC_UTILS_URL       := http://sources.buildroot.net/mmc-utils/mmc-utils-$(MMC_UTILS_URL).$(MMC_UTILS_SUFFIX)
 38 MMC_UTILS_SOURCE    := $(SRCDIR)/$(MMC_UTILS).$(MMC_UTILS_SUFFIX)
 39 MMC_UTILS_DIR       := $(BUILDDIR)/$(MMC_UTILS)
 40 MMC_UTILS_LICENSE   := GPL-2.0 AND BSD-3-Clause
                \end{verbatim}
        \end{lstlisting}
\end{itemize}
    \item[Telair International GmbH cp790760 sofware] refer
        to~\fullref{cha:software} for more details about changes.
    \item[Linux Kernel configuration] TODO
    \item[Platform configuration] TODO
    \item[PTXdist configuration] TODO
\end{description}

\footnotetext[1]{The subfolders organize patches according to different hardware
boards TQ Systems offers.}
\footnotetext[2]{See:~\fullref{part:telairpatches}}
\footnotetext[3]{See:~\fullref{part:telairpatchessources}}
