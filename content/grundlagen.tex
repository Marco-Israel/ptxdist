\chapter{Basics and general information}%
\label{cha:Basics and general information}


\section{Install PTXdist and how to use it - Building the Linux distribution. }%
\label{sec:Building the Linux distribution}

\begin{description}
    \item[To build the Linux distribution] the framework \textit{PTXdist} is
        used. Refer to \cite{PTXdist} to get an introduction to PTXdist.
    \item[How to install PTXdist] and get the framework running, refer to
        \cite{install_ptxdist}. This describes how to install the
        \textbf{right cross compiler version} as well as how to configure
        PTXdist and the BSP, provided by TQ \cite[TQ BSP]{BSP_119}. Its
        important that the cross compiler is compatible to the used Linux kernel
        as well as the other sources provided by the BSP and PTXdist
        \textit{rules} (some kind of \gls{metadaten} describing how and were to
        download and install some sources and packets).
    \item[How to use PTXdist] refer to \cite[PTXdist manual]{ptxdist_manual}
\end{description}


\section{Git}%
\label{sec:Building the Linux distribution}
Git is a distributed as well as central version control system which only saves
changes to a file instead of copies. Switching between development versions is
easily this way. Additional Git has a lot of other features compared to other
\glspl{VCS} systems. Git was developed 2005 by Linus Torvalds.
\\
Following are links to tutorials and description which
descries what git is, what its benefits are  and how you can use it.


\begin{description}
    \item[Git command referance] \cite{gitdoc}
    \item[Open PDF book] provided by the git homepage: \cite{Chacon2014}. I also
        holds an command index on the last side like the online documentation
        mentioned out above. Thanks to pdf and linkage you
        can directly jump to the descriptions with examples and well designed
        pictures. Below is a video series wich shows some of the examples and
        pictures inside this book.
    \item[Video tutorial seriers] by an English speaker:
        \cite{gitvideotutorial_en}. He orients on the book mentioned out above
        and gives examples and pictures you find inside this books.
    \item[Video tutorials] (English) by an German speaker:
        \cite{gitvideotutorial_de_en}
    \item[Cheat-Sheet] gives an overview with a short descriptions of the most
        used git commands and git commands flags: \cite{git_cheat_sheet}. You
        can download it as PDF. The website also give some practical examples
        with command line pictures to some of the commands and workflows.
    \item[Picture of states] gives an overview of the local and remote
        repositories philosophic behind git and shows which commands are
        needed in which case:
        \cite{git_picture}
\end{description}
