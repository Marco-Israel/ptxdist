\chapter{Basics and general information}%
\label{cha:Basics_and_general_information}


\section{About PTXdist}%
\label{sec:linux_distribution_build}

How to install PTXdist, how to use it and how to to build the final Linux
distribution configured by use use of PTXdist is in detail descried  online on
websites. This section provides the information where to find online
descriptions compatible to the tqma335x chip.

\begin{description}
    \item[To build the Linux distribution] the framework \textit{PTXdist} is
        used. Refer to \textbf{\cite{ptxdist} } to get an introduction to
        PTXdist.

    \item[How to install PTXdist] and get this framework running, is descried in
        \textbf{\cite{tq_bsp119_install}}. You find here how to setup your
        build host (requirements to the building host) and how to install the
        \textbf{right} (compatible) \textbf{cross compiler version}.

        \textbf{ATTENTION:} You should read the \textbf{WARNINGS} from TQ
        carefully to get the right ordering of installation and setup.

        \textbf{ADDITIONALLY} you should take a look into the
        \textbf{\cite[PTXdist general installation guide]{ptxdist_install} }
        to get information not described by TQ and to realize what is spacial
        to TQ.

    \item[How to setup TQ Systms BSP] for the base chip (like tqma335x)
        is described in short by TQ systems. Refer to
        \textbf{\cite{tq_bsp119_configurationi}} and follow the instructions to
        setup the project environment.

    \item[How to use PTXdist] you find in the
        \textbf{\cite[PTXdist user manual]{ptxdist_manual}}. Here the chapter
        \textbf{\#first-steps-with-ptxdist} is for you of interest. You can also
        take a look into \cite{ptxdist_developer} if you have more special
        questions, e.g. \textit{like how to reconfigure and recompile the linux
        kernel via PTXdist} or into the \cite[PTXdist FAQ list]{ptxdist_faq}
        (which of course is very short).

    \item[How to extend PTXdist] is descried motley in the
        \textbf{\cite[developer guide]{ptxdist_developer}}. Some extend
        information and \gls{FAQacr} are descried in
        \textbf{\cite{ptxdist_workflows}}, by other websites in the internet
        or if you write your supplier or like TQ Pengutronix an e-mail.

    \item[Background how PTXdist works] is described in the \cite[PTXdist Users
        Manual]{ptxdist_manual} in the chapter \textbf{\#how-does-it-work}.
\end{description}


\section{Git}%
\label{sec:git}
Git is a distributed as well as central version control system which only saves
changes to a file instead of copies. Switching between development versions is
easily this way. Additional Git has a lot of other features compared to other
\glspl{VCS} systems. Git was developed 2005 by Linus Torvalds.

Following are links to tutorials and description which descries what git is,
what its benefits are  and how you can use it.  \\



\begin{description}
    \item[Git command referance] \cite{gitdoc}
    \item[Open PDF book] provided by the git homepage: \cite{Chacon2014}. I also
        holds an command index on the last side like the online documentation
        mentioned out above. Thanks to pdf and linkage you
        can directly jump to the descriptions with examples and well designed
        pictures. Below is a video series wich shows some of the examples and
        pictures inside this book.
    \item[Video tutorial seriers] by an English speaker:
        \cite{gitvideotutorial_en}. He orients on the book mentioned out above
        and gives examples and pictures you find inside this books.
    \item[Video tutorials] (English) by an German speaker:
        \cite{gitvideotutorial_de_en}
    \item[Cheat-Sheet] gives an overview with a short descriptions of the most
        used git commands and git commands flags: \cite{git_cheat_sheet}. You
        can download it as PDF. The website also give some practical examples
        with command line pictures to some of the commands and workflows.
    \item[Picture of states] gives an overview of the local and remote
        repositories philosophic behind git and shows which commands are
        needed in which case:
        \cite{git_picture}
\end{description}
