\chapter{Locations}%
\label{cha:locations}

\section{Project relevant files inside the PTXdist build machine}%
\label{sec:relevant_files}


Here is brave overview over the locations where to find recent files belongs to
this project and the Telair International GmbH inside the Virtual Box Linux
machine and inside the PTXdist project.

\begin{description}

    \item[The PTXdist project itself] is located inside the \gls{VM} at:

        \begin{alltt}
        \textit{/ptxdist/projects/cp790760}
        \end{alltt}

        \textbf{This location is the \textit{project-root}} as it will be called
        further on inside this documentation.

    \item[Source files downloaded by PTXdist] like the Linux kernel or software
        packets can be found in:

        \begin{alltt}
        \textit{/ptxdist/sources}
        \end{alltt}

        This folder could be
        used as a global source folder for other projects if you are using this
        VM again. Its also possible to locate \textit{Yocto} to it.

    \item[The cp790760 software sources] are located in:

        \begin{alltt}~\label{part:location_cp790760_software}
        \textit{<project-root>/local\_sources/telair-cp\_790760\_1 }
        \end{alltt}

        PTXdist is compiling this sources during the build project and will
        place it into the SDCard rootfs.

    \item[The cp790760 data files] like pictures or configuration files, are
        located in:
        \begin{alltt}
        \textit{<project-root>/local\_sources/
        \qquad\(\hookrightarrow\) telair-partion-files/partion-data/files}
        \end{alltt}
        This are the files which will be placed in the so called \textit{data}
        partion on the SDCard.

    \item[The cp790760 log files] are located in:
        \begin{alltt}
        \textit{<project-root>/local\_sources/
        \qquad\(\hookrightarrow\) telair-partion-files/partion-log/files}
        \end{alltt}
        This files are old log files. This mechanism to copy old log files is
        only if you like to update the system and want to hold the old logs.
        Otherwise the running system will create new files after the systems
        boots the first time.

    \item[The Telair specific hardware description] which is needed by the Linux
        Kernel and the software is described in a so called \gls{DTS} file.
        This description is located in a file called:

        \begin{alltt}
        \textit{\textbf{<project-root>/local\_src/telair-patches/linux-4.4/
        \qquad\(\hookrightarrow\) /new/arm/boot/dts/telair-am335x.dts}}
        \end{alltt}
        This plain text description is compiled by PTXdist to an binary file
        called \gls{DTB} and is finally located inside the \textit{final image}
        location below on site~\pageref{part:location_final_image}.

    \item[Other rootfs files] like scripts using by he cp790760 software are
        located in:

        \begin{alltt}
        \textit{<project-root>/local\_sources/
        \qquad\(\hookrightarrow\) telair-partion-files/partion-rootfs/files}
        \end{alltt}
        This files will be flashed to the SDCard rootfs partion.

    \item[Sources of patched (updated) files] created by the Telair
        International GmbH to the software packets, the Linux kernel or
        bootloader, can be found in:

        \begin{alltt}
        \textit{<project-root>/local\_sources/telair-patches}
        \end{alltt}

        Here you find the \textit{old} and \textit{new} files inside the folders
        named corresponding to the software packets to patch.
    \item[The \textit{patches} itselfe] created by the Telair International GmbH

        \begin{alltt}
        \textit{<project-root>/patches/<packet-to-patch>/telair/}
        \end{alltt}

        This will be executed by PTXdist during build automatically.

        \textbf{ATTENTION}: You need to \textit{activate} the patches in a file
        called \textit{series.MBa335x} if you like to create new ones or rename
        the existing ones. This file defines the order in which patches should
        be preformed.

    \item[Final images] are located in:
        \begin{alltt}~\label{part:location_final_image}
        \textit{<project-rootl/platform-MBa335x/images}
        \end{alltt}
    \item[Scripts] like to \textbf{format} and \textbf{flash} the SDCard are
        located inside:
        \begin{alltt}
        \textit{<project-root>/tools/telair}
        \end{alltt}
       %Inside <project-root> you find an \gls{symlink} pointing to the tools.

    \item[Linux Kernel and PTX project configurations]are located in:
        \begin{alltt}
        \textit{<project-root>/configs/platforms/tqma335x/mba335x/}
        \end{alltt}
        Here you find the:
        \begin{itemize}
            \item Linux kernel configuration in a file called
        \textit{kernelconfig-4.4}

    \item PTXdist BSP platform configuration in a file
        called \textit{platformconfig} This is the
        \textbf{project configuration}
         \end{itemize}
        The \textbf{PTXdist general configuration} can be found in a file called
        \newline \textit{<project-root>/configs/systems/qt/ptxconfig}

         How to edit this files by a call to \textit{ptxdist} is documented in
         the \textit{First steps to PTXdist} chapter founded \newline
         in~\cite[PTXDist online documentation]{ptxdist_docu}

        You can also find symbolic links pointing to the used configurations
        inside <project-root>.

    \item[PTXdist install directory] is located inside the build host in:
        \begin{alltt}
        \newline \textit{/usr/local/lib/ptxdist-2016.04.0/}
        \end{alltt}
        There you find all general rules.make which are predefined be
        Pingutronix / PTXdist to build common packets and the Linux kernel
        itself. Only rules.make for additional software added by TQ Systems
        additionally are inside the BSP \textit{rules} directory.

    \item[Bash, VIM, Git and other Linux build host configuration] are located
        under the user \textit{ptxdist home} directory (\~/). Take a look into
        this files to get information how the virtual Linux build host is
        configured.
\end{description}

