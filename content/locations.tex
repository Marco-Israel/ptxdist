\chapter{Locations}%
\label{cha:locations}

\section{Project relevant files inside the PTXdist build machine}%
\label{sec:Where to find project relevant files inside the PTXdist build machine}



Here is brave overview over the locations where to find recent files belongs to
this project and the Telair International GmbH inside the Virtual Box Linux
machine and inside the PTXdist project. \\

\begin{description}
    \item[The PTXdist project itselve] is located inside the virtual box VM
        under  \newline
        \textit{/ptxdist/projects/cp790760}. \\
        \textbf{This location is the
            \textit{project-root}} as it will be called further on inside this
        documentaion.

    \item[Source files downloaded by PTXdist] like the Linux kernel or software
        packets can be found in \textit{/ptxdist/sources}.

        This folder could be
        used as a global source folder for other projects if you are using this
        VM again. Its also possible to locate \textit{Yocto} to it.

    \item[The cp790760 software sources] are located in \newline
        \textit{<project-root>/local\_sources/telair-cp\_790760\_1}.

        PTXdist is compiling this sources during the build project and will
        place it into the SDCard rootfs.

    \item[The cp790760 data files] like pictures or configuration files, are
        located in \newline
        \textit{<project-root>/local\_sources/telair-partion\_data\_files}. This
        are the files which will be placed in the so called \textit{data}
        partion on the SDCard.

    \item[The cp790760 log files] are located in \newline
        \textit{<project-root>/local\_sources/telair-partion\_log\_files}. This
        files are old log files. This mechanism to copy old log files is only
        if you like to update the system and want to hold the old logs.
        Otherwise the running system will create new files after the systems
        boots the first time.

    \item[Other rootfs files] like scripts using by he cp790760 software are
        located in
        \textit{<project-root>/local\_sources/telair-partion\_rootfs\_files}.
        This files will be flashed to the SDCard rootfs partion.

    \item[Patched (updated) files] created by the Telair International GmbH to
        the software packets, the Linux kernel or  bootloader, can be found in
        \newline \textit{<project-root>/local\_sources/telair-partion\_log\_files}.
        Here you find the \textit{old} and \textit{new} files inside the folders
        named corresponding to the software packets to patch.
    \item[The \textit{patches} itselfe] created by the Telair International GmbH
        are located in
        \textit{<project-root>/patches/<software-packet-to-patch>/telair/}. \\

        \textbf{ATTENTION}: You need to \textit{activate} the patches in
        the so called \textit{series.MBa335x} series file if you like to create
        new ones or rename the existing ones.

    \item[Final images] are located in
        \textit{<project-rootl/platform-MBa335x/images}

    \item[Scripts] like to \textbf{format} and \textbf{flash} the SDCard are
        located inside \newline \textit{<project-root>/tools/telair}. Inside
        <project-root> you find a symlink pointing to the tools.

    \item[Linux Kernel and PTX project configurations]are located in \newline
        \textit{<project-root>/configs/platforms/tqma335x/mba335x/}. Here you
        find the
        \begin{itemize}
            \item[Linux Kernelconfig] in a file called \textit{kernelconfig-4.4}
            \item[PTX BSP platform configuration] in a file called
                \textit{platformconfig}. This is the \textbf{project
                    configuration}.
        \end{itemize}

        The \textbf{PTXdist general configuration} can be found in a file called
        \newline \textit{<project-root>/configs/systems/qt/ptxconfig}.\\

         How to edit this files by a call to \textit{ptxdist} is documented in
         the \textit{First steps to PTXdist} chapter founded in \cite[PTXDist
         online documentation]{ptxdist_docu}.

        You can also find symlinks pointing to the used configurations inside
        <project-root>.

    \item[PTXdist install directory] is located inside the build host in
        \newline \textbf{/usr/local/lib/ptxdist-2016.04.0/}. There you find all
        general rules.make which are predefined be Pingutronix / PTXdist to
        build common packets and the linuxkernel itself. Only rules.make for
        additional software added by TQ Systems additionally are inside the BSP
        \textit{rules} directory.

    \item[Bash, VIM, Git and other Linux build host configuration] are located
        under the user \textit{ptxdist home} directory (\~/). Take a look into
        this files to get information how the virtual Linux build host is
        configured.

\end{description}

