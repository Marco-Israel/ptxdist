\chapter{Images}%

\footnotetext[1]{See \fullref{part:location_cp790760_software}}
\footnotetext[2]{see \fullref{cha:locations}}
\footnotetext[3]{See \fullref{sec:build_image_script}}


So called \textit{images} are the final output of PTXdist that what you need to
install or flash to your SDCard, or internal SOC memory. This chapter descries
the currently available images inside this project and how to build
them by use of PTXDist.


\section{Build the images.}%
\label{sec:final_images_generation}
To build images first change to you <projectroot>~\footnotemark[1] directory.

Afterwards run:

\bigbreak%
\begin{lstlisting}[caption=Building of all Images]
user@machine:  ptxdist images
\end{lstlisting}

Beside from this you can build only one of all images explicitly. For
instance to (re-) build the rootfs partion image run:

\bigbreak%
\begin{lstlisting}[caption=Example: Build the RootFS automatically]
user@machine:  ptxdist image rootfs
\end{lstlisting}

\textbf{NOTE} the difference between \textit{image} and \textit{image\textbf{s}}
\bigbreak%

Other of the images below~\footnotemark[2] can be (re-) buided the
same way by simple replacing the \textit{target} keyword \textit{ptxdist}.

\section{Final Images}%
\label{sec:final_images}

\subsection{Howl disk images}
\label{sub:howl_disk}

The following images are complete SDCard images which can be flashed to the
SDCard.


\begin{itemize}
    \item sd.hdimg: is a bootable image primarily for SD Card.
        This one can be also used for memory chip.
    \item boot-mlo.vfat: fat filesystem with MLO, u-boot and Linux-kernel
\end{itemize}

\textbf{ATTATION:} This image types seems not to work if you flash it.
\textbf{Please use the \textit{generate script} in the tools telair
location~\footnotemark[2]}.

This will generate an image called:
\begin{itemize}
    \item telair-cp790760.img:  An bootable image preinstalled with content in
        all partions needed by this project.
\end{itemize}

More information about this script, you find in \fullref{sec:build_image_script}

\subsection{Bootloader images}%
\label{sec:bootloader_images}

The following Bootloader images are created. They needs to copy on the SDCard
or MMc memory into the \textbf{boot partion}. This can be done automatically
by using one of the tools telair scripts~\footnotemark[3]

\begin{itemize}
    \item u-boot.bin
    \item u-boot.img boot loader image for SD Card/eMMC
    \item MLO sendodary bootloader, also called \gls{SPL}
\end{itemize}


\section{Liunx images}%
\label{sec:liunx_images}

The following Linux images are created:

\begin{itemize}
    \item linuximage: An zImage generated by PTXdist with this name. It needs to
        be renamed to \textit{zImage}
    \item zImage: The renamed linuximage. It needs to be copied into the
        partion or memory section \textit{boot}.
    \item uImage: The zImage with an addition header needed by u-boot
    \item am335x-mba335x.dtb: device tree blob for the Telair International GmbH
        \newline \gls{PCBA} hardware.
\end{itemize}


\subsection{Root fs images}%
\label{sec:rootfs_imges}

\begin{itemize}
    \item root.ext2: The howl content to the rootfs partion or memory section
    using the ext2/3/4 filesystem.
\end{itemize}


