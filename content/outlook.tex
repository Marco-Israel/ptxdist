
\chapter{Outlook into the feature}%
\label{cha:ausblick}


\section{U-Boot working with NFS and TFTP}%
\label{sec:uboot_nfstftp}

For further development it would be useful to get an TFTP and NFS server
running. Therefor an Ethernet interface is needed on an development board but
would decrease the development time significantly.

In this case, the bootloader environment needs to be modified to work with your
\gls{TFTP} server and your \gls{NFS} server. Here are some steps needs to be
preformed to get it running

\newpage

\begin{lstlisting}[gobble=4, language=bash,numbers=left,caption=Configure
bootboloader (Barebox or U-Boot) to use NTFS and TFTP.]
         #The bootloader should try to load any image using TFTP
    setenv autoload no

        #The IP of the TFT-Server host. Like 192.168.100.1
    setenv serverip <serverip>

        #Own machine IP address. Like 192.168.100.10
    setenv ipaddr <ipaddr>

        #Networmask bothe machines are in. Like 255.255.255.0
    setenv netmask <netmask>

        #Path to root folder which should be shared
        #ATTATION   NFS share has to set in /etc/exports
        #ATTATION   NFS-Server has to run on the host
    setenv rootpath <rootpath>

        #name of devicetree file to be downloaded
    setenv dtb <dtb>

        #name of the Linux kernel image to be downloaded
    setenv kernel <kernel>

        #Direct start load and start the kernel
        #NOTE   Setting is not needed and not depended to TFTP/NFS
    setenc autostart on

\end{lstlisting}

\newpage

\section{QEMU}%
\label{sec:QEMU}
To decrease the development time a split software from hardware bugs, it would
be useful to simulate the hardware by QEMU.\@ This would make the use of an NFS
and TFTP server possible easy by adding an virtual network adapter.
