\chapter{SDCard partion layout, format and installation}%
\label{cha:sdcard}

To get the SDCard running following steps are required described in the
next sub chapters.

\begin{enumerate}
    \item Create a partion layout.
    \item Format the SDCard partions.
    \item Install the Files into the Partions
\end{enumerate}


\section{Partion layout creation}%
\label{sub:sdcard_layout}
The SDCard needs to have the following four partions. The ordering is important.

\begin{description}
    \item[1. Boot] The first partion. Holding the Bootloaders (u-Boot, MLO),
        the zImage and uImage (both are compressed Linux kernels, ones needed
        for the u-boot) and the Linux device tree blob.
    \item[2. RootFS] The second partion. This is holding the complete Linux
        \gls{rootfs}.
    \item[3. Data] The third partion. This holds the cp790760 data files needed
        by the application. Like pictures or configuration files.
    \item[4. Log] The fourth partion. This partion is holding all log files
        created by the cp790760 application and the Linux kernel.
\end{description}


\section{Formatting the SDCard}%
\label{sub:sdcard_format}
The SDCard partions need to have the follow format types in the table below.

\begin{table}
\centering
{\rowcolors{3}{blue!80!white!50}{blue!70!white!40}
    \begin{tabular}{ | c | l | c | }
        \hline
        \rowcolor{lightgray} \multicolumn{3}{|c|}{Partion formats} \\
        \hline
        \textbf{ID} & {\bfseries Partion name} & {\bfseries Partion format} \\
        \hline
         1 & Boot & vFat \\
         2 & RootFs & ext4 \\
         3 & Data & vFat \\
         4 & Log & vFat \\
        \hline
    \end{tabular}
}
\caption{This is a table}
\label{tab:partion_formats}
\end{table}

\section{Installing the application}%
\label{sec:sdcard_appinstall}
There have to be installed (copy) different files to to the different partions.
