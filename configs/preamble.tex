% ##################################################
% Unterstuetzung fuer die deutsche Sprache
% ##################################################
%\usepackage{ngerman}
\usepackage[ngerman]{babel}

% ##################################################
% Dokumentvariablen
% ##################################################

% Persoenliche Daten
\newcommand{\docNachname}{Israel}
\newcommand{\docVorname}{Marco}
\newcommand{\docStrasse}{Am Wickenkamp 38}
\newcommand{\docOrt}{Stemwede}
\newcommand{\docPlz}{32351}
\newcommand{\docEmail}{Marco-Israel-Consulting@gmx.de}
\newcommand{\docMatrikelnummer}{242814}

% Dokumentdaten
\newcommand{\docTitle}{Embedded Linux}
\newcommand{\docUntertitle}{mit Yocto und OpenEmbedded}
%\newcommand{\docUntertitle}{} % Kein Untertitel
% Arten der Arbeit: Bachelorthesis, Masterthesis, Seminararbeit, Diplomarbeit
\newcommand{\docArtDerArbeit}{}
%Studiengaenge: Allgemeine Informatik Bachelor, Computer Networking Bachelor,
% Software-Produktmanagement Bachelor, Advanced Computer Scinece Master
\newcommand{\docStudiengang}{}
\newcommand{\docAbgabedatum}{Dezember 2019}
\newcommand{\docErsterReferent}{}
%\newcommand{\docZweiterReferent}{-} % Wenn es nur einen Betreuer gibt
%\newcommand{\docZweiterReferent}{ZWEITER REFERENT}

% ##################################################
% Allgemeine Pakete
% ##################################################

% Abbildungen einbinden
\usepackage{graphicx}

% Zusaetsliche Sonderzeichen
\usepackage{dingbat}

% Farben
\usepackage[usenames,dvipsnames,svgnames,table]{xcolor}

% Maskierung von URLs und Dateipfaden
\usepackage{url}

% Deutsche Anfuehrungszeichen
\usepackage[babel, german=quotes]{csquotes}

% Pakte zur Index-Erstellung (Schlagwortverzeichnis)
\usepackage{index}
\makeindex
%\usepackage[xindy]{imakeidx}

% Ipsum Lorem
% Paket wird nur für das Beispiel gebraucht und kann gelöscht werden
\usepackage{lipsum}


% ##################################################
% New commands
% ##################################################
\newcommand*{\fullref}[1]{\hyperref[{#1}]{\autoref*{#1} {Seite} \pageref*{#1}{;}
\nameref*{#1} }}
%\newcommand*{\namepageref}[1]{\hyperref[{#1}]{\nameref*{#1} \pageref*{#1} }}


% ##################################################
% Seitenformatierung
% ##################################################
\usepackage[
	portrait,
	bindingoffset=1.5cm,
	inner=2.5cm,
	outer=2.5cm,
	top=3cm,
	bottom=2cm,
	includeheadfoot
	]{geometry}

% ##################################################
% Kopf- und Fusszeile
% ##################################################

\usepackage{fancyhdr}

\pagestyle{fancy}
\fancyhf{}
\fancyhead[EL,OR]{\sffamily\thepage}
\fancyhead[ER,OL]{\sffamily\leftmark}

\fancypagestyle{headings}{}

\fancypagestyle{plain}{}

\fancypagestyle{empty}{
  \fancyhf{}
  \renewcommand{\headrulewidth}{0pt}
}

%Kein "Kapitel # NAME" in der Kopfzeile
\renewcommand{\chaptermark}[1]{
	\markboth{#1}{}
   	\markboth{\thechapter.\ #1}{}
}

% ##################################################
% Schriften
% ##################################################

% Stdandardschrift festlegen
\renewcommand{\familydefault}{\sfdefault}

% Standard Zeilenabstand: 1,5 zeilig
\usepackage{setspace}
\onehalfspacing

% Schriftgroessen festlegen
\addtokomafont{chapter}{\sffamily\large\bfseries}
\addtokomafont{section}{\sffamily\normalsize\bfseries}
\addtokomafont{subsection}{\sffamily\normalsize\mdseries}
\addtokomafont{caption}{\sffamily\normalsize\mdseries}

% ##################################################
% Inhaltsverzeichnis / Allgemeine Verzeichniseinstellungen
% ##################################################

% Control table of contents, figures, listings, etc
\usepackage{tocloft}

% Punkte auch bei Kapiteln
\renewcommand{\cftchapdotsep}{3}
\renewcommand{\cftdotsep}{3}

% Schriftart und -groesse im Inhaltsverzeichnis anpassen
\renewcommand{\cftchapfont}{\sffamily\normalsize}
\renewcommand{\cftsecfont}{\sffamily\normalsize}
\renewcommand{\cftsubsecfont}{\sffamily\normalsize}
\renewcommand{\cftchappagefont}{\sffamily\normalsize}
\renewcommand{\cftsecpagefont}{\sffamily\normalsize}
\renewcommand{\cftsubsecpagefont}{\sffamily\normalsize}

%Zeilenabstand in den Verzeichnissen einstellen
\setlength{\cftparskip}{.5\baselineskip}
\setlength{\cftbeforechapskip}{.1\baselineskip}

% ##################################################
% Abbildungsverzeichnis und Abbildungen
% ##################################################

\usepackage{caption}

\usepackage{wrapfig}

% Nummerierung von Abbildungen
\renewcommand{\thefigure}{\arabic{figure}}

% Abbildungsverzeichnis anpassen
\renewcommand{\cftfigpresnum}{Abbildung }
\renewcommand{\cftfigaftersnum}{:}

% Breite des Nummerierungsbereiches [Abbildung 1:]
\newlength{\figureLength}
\settowidth{\figureLength}{\bfseries\cftfigpresnum\cftfigaftersnum}
\setlength{\cftfignumwidth}{\figureLength}
\setlength{\cftfigindent}{0cm}

% Schriftart anpassen
\renewcommand\cftfigfont{\sffamily}
\renewcommand\cftfigpagefont{\sffamily}

% ##################################################
% Tabellenverzeichnis und Tabellen
% ##################################################

% Nummerierung von Tabellen
\renewcommand{\thetable}{\arabic{table}}

% Tabellenverzeichnis anpassen
\renewcommand{\cfttabpresnum}{Tabelle }
\renewcommand{\cfttabaftersnum}{:}

% Breite des Nummerierungsbereiches [Abbildung 1:]
\newlength{\tableLength}
\settowidth{\tableLength}{\bfseries\cfttabpresnum\cfttabaftersnum}
\setlength{\cfttabnumwidth}{\tableLength}
\setlength{\cfttabindent}{0cm}

%Schriftart anpassen
\renewcommand\cfttabfont{\sffamily}
\renewcommand\cfttabpagefont{\sffamily}

% Unterdrueckung von vertikalen Linien
\usepackage{booktabs}

% ##################################################
% Listings (Quellcode)
% ##################################################

%\usepackage{listings}
%\lstset{
%	language=java,
%	backgroundcolor=\color{white},
%	breaklines=true,
%	prebreak={\carriagereturn},
% 	breakautoindent=true,
% 	numbers=left,
% 	numberstyle=\tiny,
% 	stepnumber=2,
% 	numbersep=5pt,
% 	keywordstyle=\color{blue},
%   	commentstyle=\color{green},
%   	stringstyle=\color{gray}
%}

%\newcommand{\includecode}[2][c]{\lstinputlisting[caption=#2, escapechar=, style=custom#1]{#2}<!---->}
% Language define input
%\input{configs/listings-bash.prf}
%\input{configs/listings-C.prf}
%\input{configs/listings-C++.prf}
%\input{configs/listings-Python.prf}
%\input{configs/listings-Java.prf}


\lstset{
	language=bash,
	backgroundcolor=\color{gray},
    breaklines=true,                 % sets automatic line breaking
	prebreak={\carriagereturn},
 	breakautoindent=true,
    stepnumber=2,                    % the step between two line-numbers. If it's 1, each line will be numbered
    numbersep=10pt,                  % how far the line-numbers are from the code
    rulecolor=\color{black},         % if not set, the frame-color may be changed on line-breaks within not-black text (e.g. comments (green here))
    frame=single,	                 % adds a frame around the code
 	keywordstyle=\color{blue},
   	commentstyle=\color{green},
    stringstyle=\color{gray},        % string literal style
    numberstyle=\tiny\color{black},  % the style that is used for the line-numbers
    numbers=left,                    % where to put the line-numbers; possible values are (none, left, right)
    showspaces=false,                % show spaces everywhere adding particular underscores; it overrides 'showstringspaces'
    showstringspaces=false,          % underline spaces within strings only
    showtabs=false,                  % show tabs within strings adding particular underscores
    tabsize=2,	                     % sets default tabsize to 2 spaces
    title=\lstname                   % show the filename of files included with \lstinputlisting; also try caption instead of title
}


%\lstset{
%  backgroundcolor=\color{white},   % choose the background color; you must add \usepackage{color} or \usepackage{xcolor}; should come as last argument
%  basicstyle=\footnotesize,        % the size of the fonts that are used for the code
%  breakatwhitespace=false,         % sets if automatic breaks should only happen at whitespace
%  breaklines=true,                 % sets automatic line breaking
%  captionpos=b,                    % sets the caption-position to bottom
%  commentstyle=\color{green},      % comment style
%  deletekeywords={...},            % if you want to delete keywords from the given language
%  escapeinside={\%*}{*)},          % if you want to add LaTeX within your code
%  extendedchars=true,              % lets you use non-ASCII characters; for 8-bits encodings only, does not work with UTF-8
%  firstnumber=1000,                % start line enumeration with line 1000
%  keepspaces=true,                 % keeps spaces in text, useful for keeping indentation of code (possibly needs columns=flexible)
%  keywordstyle=\color{blue},       % keyword style
%  language=Octave,                 % the language of the code
%  morekeywords={*,...},            % if you want to add more keywords to the set
%  numbers=left,                    % where to put the line-numbers; possible values are (none, left, right)
%  numberstyle=\tiny\color{gray},   % the style that is used for the line-numbers
%  showspaces=false,                % show spaces everywhere adding particular underscores; it overrides 'showstringspaces'
%  showstringspaces=false,          % underline spaces within strings only
%  showtabs=false,                  % show tabs within strings adding particular underscores
%  stringstyle=\color{black},       % string literal style
%  tabsize=2,	                   % sets default tabsize to 2 spaces
%  title=\lstname                   % show the filename of files included with \lstinputlisting; also try caption instead of title
%}
%

% ##################################################
% Theoreme
% ##################################################

% Umgebung fuer Beispiele
\newtheorem{beispiel}{Beispiel}

% Umgebung fuer These
\newtheorem{these}{These}

% Umgebung fuer Definitionen
\newtheorem{definition}{Definition}

% ##################################################
% Literaturverzeichnis
% ##################################################

%\usepackage[
%]{bibgerm}

\usepackage[backend=biber, %% Hilfsprogramm "biber" (statt "biblatex" oder "bibtex")
%style=authoryear, %% Zitierstil (siehe Dokumentation)
style=alphabetic,
%style=numeric,
%style=ieee,
natbib=true, %% Bereitstellen von natbib-kompatiblen Zitierkommandos
hyperref=true, %% hyperref-Paket verwenden, um Links zu erstellen
]{biblatex}

\addbibresource{bibtex} %% Einbinden der bib-Datei

% ##################################################
% PDF / Dokumenteninternelinks
% ##################################################

\usepackage[
	colorlinks=false,
   	linkcolor=red,
   	citecolor=green,
  	filecolor=magenta,
	urlcolor=cyan,
    bookmarks=true,
    bookmarksopen=true,
    bookmarksopenlevel=3,
    bookmarksnumbered,
    plainpages=false,
    pdfpagelabels=true,
    hyperfootnotes,
    pdftitle ={\docTitle},
    pdfauthor={\docVorname~\docNachname},
    pdfcreator={\docVorname~\docNachname}]{hyperref}



% ##################################################
% Glossrry
% ##################################################
    %\usepackage[nomain,acronym,xindy,toc]{glossaries}
    \usepackage[acronym]{glossaries}
    \setglossarystyle{listhypergroup}
    \makeglossaries
    \loadglsentries{content/glossary} %Glossary in externer Datei...

